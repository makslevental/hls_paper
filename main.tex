\documentclass[sigconf,techreport]{acmart}

\usepackage[utf8]{inputenc}
\usepackage{url}
\usepackage{color,fancyvrb}
\usepackage{mathtools}
\usepackage{caption}
\usepackage{listings}



\setcopyright{acmlicensed}
\acmDOI{00.000/000_0}
\acmISBN{000-0000-00-000/00/00}
\acmConference[ABC '00]{A Boilerplate Conference}{January 0000}{Norfolk, Virginia, USA}
\acmYear{0000}
\copyrightyear{0000}
\acmPrice{0.00}

% When "techreport" class parameter is supplied:
%   * hide reference block from the front page
%   * hide copyright note from the front page
%   * change conference info in the header of alternate pages to title
%   * add page number in the center of the footer of non-title pages
\makeatletter
\@ifclasswith{acmart}{techreport}{
	\settopmatter{printacmref=false}
	\renewcommand\footnotetextcopyrightpermission[1]{}
	\fancyhead{}
	\fancyfoot{}
	\fancyhead[L]{\shorttitle}
	\fancyhead[R]{\shortauthors}
	\fancyfoot[C]{\thepage}
	% \AtBeginDocument{The pagesize is \texttt{a5paper}.\par}
}{}
\makeatother

\setlength{\belowcaptionskip}{10pt plus 3pt minus 2pt} % Chosen fairly arbitrarily
\setlength{\abovecaptionskip}{10pt plus 3pt minus 2pt} % Chosen fairly arbitrarily

\newenvironment{mylisting}[1]
    {\def\savedcaption{\caption{#1}}%
	\captionsetup{type=lstlisting}
	\par\noindent\hrulefill
	\resetlinenumber
	\par\begin{linenumbers}
	\bf
	\fontfamily{cmtt}\selectfont
	\footnotesize
	\setlength{\parindent}{0pt}
    }
    { 
    \end{linenumbers}
	\vspace{-5pt}\hrulefill
	\savedcaption
    }

\newenvironment{mylistingnocap}
    {\par\noindent\hrulefill
	\par
	\bf
	\fontfamily{cmtt}\selectfont
	\scriptsize
	\setlength{\parindent}{0pt}
    }
    { 
	\vspace{-5pt}\hrulefill\\
    }


\newcommand\crule[3][black]{\textcolor{#1}{\rule{#2}{#3}}}

\usepackage[dvipsnames]{xcolor}
\colorlet{highlightcolor1}{white!70!green}
\colorlet{highlightcolor2}{white!70!yellow}

\usepackage{soul}
\sethlcolor{highlightcolor1}

\usepackage{textgreek}
\usepackage{stmaryrd}
\usepackage{siunitx}

% % for including other docs (like rtf stuff)
\usepackage{standalone}

\newcommand{\hlc}[2][yellow]{{%
    \colorlet{foo}{#1}%
    \sethlcolor{foo}\hl{#2}}%
}

\usepackage{lineno}


\newif\iffinal
\finalfalse


\iffinal
	\newcommand{\maxx}[1]{}
	\newcommand{\ryan}[1]{}
	\newcommand{\kyle}[1]{}
	\newcommand{\ian}[1]{}
	\newcommand{\arham}[1]{}
	\newcommand{\commnt}[2]{#2}
\else
	\newcommand{\maxx}[1]{{\textcolor{red}{ Max: #1 }}}
	\newcommand{\ryan}[1]{{\textcolor{magenta}{ Ryan: #1 }}}
	\newcommand{\kyle}[1]{{\textcolor{yellow}{ Kyle: #1 }}}
	\newcommand{\ian}[1]{{\textcolor{orange}{ Ian: #1 }}}
	\newcommand{\arham}[1]{{\textcolor{pink}{ Arham: #1 }}}
	\newcommand{\commnt}[2]{{{\color{green} \{#1\}} {\color{blue} #2}}}
\fi


\begin{document}

\title{A Framework for Counting Archived Frogs}

\author{Maksim Levental, Arham, Kaz, Ryan "the champ", Kyle, and Ian Foster}
\affiliation{%
	\institution{University of Chicago}
	\department{Department of Computer Science}
	\city{Chicago}
	\state{Illinois}
	\postcode{60637}
	\country{USA}
}
\email{{mlevental,..,foster}@uchicago.edu}

\renewcommand{\shortauthors}{Levental et al.}


\begin{abstract}
	In many experiment-driven scientific domains, such as high-energy physics and X-ray crystallography, high sample rates necessitate low latency near-sensor data processing.
	Thanks to a recent profusion in high-level frameworks, and example models, Deep Neural Network (DNN) models have been investigated for such use cases.
	Despite such investigations, few DNNs have been deployed in practice, owing to the inability of deployment targets (hardware accelerators) to meet hard latency and collocation constraints.
	Here we present a case-study of translating/deploying a particular X-ray crystallography DNN model to an alternative platform, namely Field Programmable Gate Arrays.
	We discuss some of the currently available workflows, toolchains, and their advantages and disadvantages.
	Further we discuss an application specific design methodology (specific to this model) to general purpose tools.
	Our approach achieves lower latency than any of the alternatives at the cost of generalizability, achieving 3µs/333KHz.
	Finally, we discuss extensions to our methodology that would enable generalization without any sacrifice in performance.
\end{abstract}


%    \begin{CCSXML}
%    <ccs2012>
%    <concept>
%    <concept_id>
%        10002951.10003227.10003392</concept_id>
%        <concept_desc>Information systems~Digital libraries and archives</concept_desc>
%        <concept_significance>500</concept_significance>
%        </concept>
%        <concept>
%        <concept_id>10002951.10003260</concept_id>
%        <concept_desc>Information systems~World Wide Web</concept_desc>
%        <concept_significance>500</concept_significance>
%        </concept>
%        </ccs2012>
%    \end{CCSXML}

\ccsdesc[500]{Information systems~Digital libraries and archives}
\ccsdesc[500]{Information systems~World Wide Web}

% Comment the above block out to hide CCS Concepts or update as per https://dl.acm.org/ccs/ccs.cfm


\keywords{Memento, Web Archiving, Frogs}


\maketitle

\section{Introduction}\label{sec:introduction}
Certain areas of science perform experiments that achieve extremely high data rates; 
well-known examples in particle physics are the Compact Muon Solenoid (CMS) and Toroidal Apparatus (ATLAS) experiments at the Large Hadron Collider (LHC).
These experiments observe new collision events every 25~ns (i.e., at a rate of 40~MHz) and for which \emph{triggers} must report detection of ``interesting'' events within, at most, 10~\textmu s ~\cite{pmlr-v42-glig14}.
Similarly, X-ray crystallography employs high-energy diffraction microscopy (HEDM) techniques, which can sample at up to 1~MHz~\cite{doi:10.1063/5.0006531} (and also must trigger within 10~\textmu s).
Such high data rate experiments face the challenge of either processing the data in situ (colocated with the scientific apparatus) or buffering/caching for later retrieval and post-processing.
For example, if CMS and ATLAS were to capture all collision events, they would produce approximately 40 terabytes per second~\cite{BORK2021100619}.
Thus, any improvement in the real-time, near-sensor, processing capabilities of experiment infrastructure can dramatically reduce the burden on downstream infrastructure and compute, in addition to accelerating the pace of experiment design and scientific discovery.

% \ian{I found the next text hard to follow because it doesn't really explain WHY DNNs are being considered in the first place, which is that due to their ability to perform specialized data processing tasks with high efficiency, appropriately trained DNNs are being used to implement analyses that could not otherwise be performed online, at least not without custom silicon.
% Maybe say that first (using BraggNN example to illustrate the large speedups possible: give number), and then explain that particularly at very high data rates, even DNNs cannot deliver required performance on ``conventional'' hardware (GPUs or other such DNN accelerators), which furthermore have large form factors that preclude colocation with sensors.}

In general, classical, physics-based, methods for these data processing tasks cannot meet requisite latency constraints~\cite{Sharma:rw5009}.
Deep neural networks, functioning as learned but efficient approximations of the physics-based methods, and effective in many other academic and commercial domains, have recently been considered for these use cases~\cite{Guest:2018yhq}.
For example, BraggNN~\cite{Liu:fs5198}, a DNN aimed at efficiently identifying Bragg diffraction peaks, achieves detection rates 200x in excess of classical pseudo-Voigt methods, with high accuracy.
Still, as of yet DNN models have not seen wide adoption in these high data-rate experimental domains.
This is due to the limitations imposed by the hardware platforms on which they can typically be deployed: commodity general purpose processors (CPUs), graphics processing units (GPUs) and other, exotic, DNN accelerators.
Primarily, such accelerators cannot meet the hard real-time latency constraints, and secondarily they cannot be easily colocated with complex sensing apparatuses.
Case in point: BraggNN, despite having been shown to have high speedup over the classical pseudo-Voigt peak fitting methods, making determinations in approximately 700\textmu s, still falls far short of the 1\textmu s target for handling the 1 MHz sampling rates necessary of HEDM experiments.
In addition, the current implementation of BraggNN, deployed to either a datacenter class GPU such as a NVIDIA V100, or even a workstation class GPU such as a NVIDIA RTX 2080Ti, has no practicable means to being deployed at the edge, i.e., adjacent or proximal to the high energy microscopy equipment.

This work considers a potential alternative deployment platform for deploying BraggNN, namely Field Programmable Gate Arrays (FPGAs), for the purpose of enabling edge Bragg peak detection in data collected in the course of HEDM.
FPGA represent a fabric of hardware units, such as multiplexers (MUXs), lookup tables (LUTs), block RAMs (BRAMs), flip-flops (FFs), and digital signal processors (DSPs), connected together by programmable interconnects.
FPGAs present an appealing alternative data processing platform for low latency scientific use cases for three reasons:
\begin{enumerate}
	\item FPGAs can be configured to implement arbitrary functions with minimal \emph{abstraction cost};
	\item FPGAs can be reconfigured an arbitrary number of times;
	\item FPGAs can be packaged in such a way that they can be colocated with scientific apparatus.
\end{enumerate}
By abstraction cost we mean two aspects of conventional hardware and software:
\begin{enumerate}
	\item the runtime cost of software infrastructure, such as process scheduling and memory management, that enables multitenancy in general purpose compute systems;
	\item the performance and efficiency costs incurred due to compute architecture that aims to be pareto optimal for a wide range of compute tasks; for example, branch prediction circuits in CPUs and fixed width\footnote{Here we refer to the availability of a fixed set of IEEE 754 floating point precisions (usually only single, 32 bit, and double, 64 bit, precision), rather than fixed precision.} floating point units in GPUs.
\end{enumerate}
To wit: the LHC currently employs FPGAs (in combination with application specific integrated circuits) for their Level-1 triggering system.

But FPGAs are not a free lunch; there are two principal challenges in deploying DNNs to FPGAs.
The first is in translating existing representations of DNNs to Register Transfer Level (RTL) representations, which can be used to configure FPGAs.
Note, DNNs are typically represented in very high-level deep learning frameworks (such as PyTorch and Tensorflow) that abstract away all implementation details. 
For example, a convolution operation, specified as \texttt{torch.nn.Conv2d(1, 16, 3)}, indicates (prima facie) neither the precise data path of the inputs nor the control flow of the executor; high-level DNN frameworks aim to be \emph{declarative}, intentionally eschewing such low-level details.
Thus it becomes necessary to \emph{lower} the high-level representation to a lower level representation, which explicitly represents the data path and control flow; effectively one must compile the DNN.
But compilation necessarily implies some instruction set architecture, i.e. some basic set of data path (e.g., \texttt{mov}, \texttt{push}) and control flow (e.g., \texttt{jmp}, \texttt{jne}) primitives,
whereas FPGAs support no such primitives.
Indeed, for typical use-cases of FPGAs, the development methodology involves a great deal of hand-written or ``hand-generated''~\cite{nikhil2004bluespec} design of the primitive components (adders, multipliers, buses, etc.), a methodology dramatically distinct from conventional software design in general, and DNN design in particular.
Thus, deploying to FPGA entails reimagining a DNN model as compute architecture unto itself, including sophisticated considerations such as operation scheduling, register pipelining, and wire delay.
Recently, with the advent of DNN compiler technologies, such as Multi-level Intermediate Representation (MLIR)~\cite{https://doi.org/10.48550/arxiv.2002.11054}, and supported by advanced High-Level Synthesis (HLS) tools, it has become possible to produce RTL representations of DNN models, with minimal intervention on the part of the user.
This work explores such a design methodology, including some of the currently available tools that aim to support an ``end-to-end flow'', i.e., a design process that takes as input a high-level representation of a DNN and produces synthesizable RTL.

The second principal challenge to deploying DNNs to FPGAs are more fundamental; current generation FPGAs \maxx{TODO: <cannot do what? cannot take the heat? cannot route?>}.
This work investigates techniques for mitigating issues related to these hardware limitations, including approximations that reduce circuit complexity, and alternative scheduling methods. 

In summary we deploy BraggNN to FPGA by performing a series of progressive lowerings, starting from a high-level representation (a PyTorch model) and culminating in synthesizable RTL (i.e., a representation that can be directly mapped to FPGA hardware).
We show that under certain assumptions, our approach produces inference latencies lower than that of any existing tool, achieving a peak end-to-end latency of 3 µs/333 KHz.
This latency represents a 200x improvement over the GPU implementation of BraggNN and is only a factor of three distant from the ultimate 1 µs/1 MHz latency target.
Our work includes a survey of existing general purpose tools aimed at performing such lowerings, as well as our own novel approach.
Thus, the primary contributions of this paper are:
\begin{enumerate}
	\item A comprehensive discussion of MLIR, as state-of-the-art DNN compiler technology, and its relevance to FPGA design;
	\item A description of generally applicable techniques for translating a DNN model from a high-level representation to a RTL representation;
	\item An application of the aforementioned techniques to the case of BraggNN.
\end{enumerate}
The remainder of this article is structured as such:
\begin{enumerate}
	\item Background on DNN compiler technology, high-level synthesis, and FPGA implementation;
	\item Our approach as compared to existing tools, with particular on difficulties faced (and overcome);
	\item An evaluation of our approach, as compared to existing tools, in terms of the latency and resource usage of BraggNN;
	\item High priority goals, as we see them, for making this flow more ergonomic.
\end{enumerate}

\section{Background}\label{sec:background}
\subsection{BraggNN}\label{subsec:braggnn}

TODO: stuff about BraggNN

\subsection{Translating DNNs}\label{subsec:translatingdnns}
Our design methodology for BraggNN winds its way through several levels of abstraction and tooling:

\begin{enumerate}
	\item A conventional PyTorch representation;
	\item A JIT traced representation called TorchScript;
	\item Several, successively lower-level, MLIR representations;
	\item A LLVM IR representation;
	\item A RTL representation.
\end{enumerate}

We quickly review the relevant concepts of each level of abstraction.

\subsubsection{PyTorch/TorchScript}\label{subsec:pytorch}

Typically DNN models are represented in terms of high-level frameworks implemented within general purpose programming languages.
Such frameworks are widely used because of their ease of use and large library of example implementations of various DNN model architectures.
Two such frameworks are TensorFlow and PyTorch.
BraggNN is implemented within PyTorch.
DNNs are developed within PyTorch using the \emph{define-by-run} methodology (also known as \emph{eager mode}).
Using this methodology, the developer writes conventional Python code that describes the sequential execution of the high-level operations comprising the model, and the dataflow graph (for purposes of backprop/autodiff) is defined/constructed at runtime.
With respect to the developer, this define-by-run methodology, which is not unique to PyTorch, enables fast iteration at development time, at the cost of some runtime performance (versus static specification methodologies).

With respect to the kind of program analysis necessitated by our attempted translation, there is another cost to define-by-run DNN specification: the dataflow graph (DFG) is never fully materialized\footnote{``...instead, every intermediate result records only the subset of the computation graph that was relevant to their computation.''\cite{paszke2017automatic}} and the control flow graph (CFG) is difficult to extract from the semantics of the general purpose language (Python in the case of PyTorch).
The PyTorch organization, having recognized these issues (in the course and context of their own deployment projects), in recent years has implemented a Single Static Assignment (SSA) intermediate representation (IR), called TorchScript (TS) and concomitant tracing mechanism (colloquially referred to as the TS JIT compiler) to produce TS from conventionally defined PyTorch models.
The exact operation of this tracing mechanism is beyond the scope of our work\footnote{\url{https://github.com/pytorch/pytorch/wiki/PyTorch-dispatcher-walkthrough}}, but two of its limitations, as they pertain to our work, merit discussion.
Firstly, much like other JIT compilers, the TS JIT does not\footnote{In fact it does (\mintinline{mlir}{prim::If}, \mintinline{mlir}{prim::While}) but you have to write those yourself.} effectively support control flow in the DNN model specification.
In reality, even if it did, we would still be incapable of supporting such dynamism since FPGAs do not (currently) effectively support runtime reconfiguration\cite{reconfigfpga}.
Secondly, TS IR does not always produce fully refined (i.e., shaped) tensor types.
That is to say, tensor shapes, as they appear in the TS IR, are either absent or possess symbolic dimensions\cite{10.1145/3211346.3211348}; for much the same reason as in the case of control flow, our approach necessitates fully known tensor shapes, and for this we rely on explicit annotation.

In general neither of these limitations is a serious impediment to deployment; \commnt{run the test}{only x/y models in the standard benchmark torchbench} exhibit either or both types of dynamism, and our target model, BraggNN, exhibits neither.
\begin{figure*}
	\includegraphics[width=\textwidth]{figures/BraggNN}
	\caption{This is a placeholder.}
\end{figure*}
Lowering from PyTorch to TS IR allows us to perform many useful analyses and transformations on BraggNN that would be extremely difficult or impossible on the original representation;
basic optimizations like dead code elimination and constant propagation are supported by TS's graph rewriting functions.
In addition, PyTorch also supports at least two kernel fusion\cite{10.1145/2688500.2688521} tools, proprietary (NNC fuser) and third party (nvfuser).
Such transformations are critical for achieving peak performance on CPUs and hardware accelerators alike but for our purposes, deployment to application specific hardware, we require a broader collection of transformations.
To this end, we turn to a recent addition to the compiler toolchain ecosystem.

\subsubsection{MLIR}\label{subsec:mlir}

Multi-level Intermediate Representation\cite{https://doi.org/10.48550/arxiv.2002.11054} (MLIR) is a new approach to building reusable and extensible compiler infrastructure.
MLIR is composed of a set of \emph{dialect} IRs, subsets of which are fully mutually compatible outright or by way of translation/legalization.
The various dialects aim to capture and formalize the semantics of compute intensive programs at varying levels of abstraction, as well as namespace related sets of IR transformations/optimizations (called \emph{passes}).
Our entrypoint into this compiler stack is the Torch dialect\cite{torch-mlir}, a high-fidelity mapping from TS IR to MLIR native IR, that provides for us the necessary shape refinement mentioned above, and various other translations that reduce the impedance mismatch between TS and MLIR.

While Torch dialect acts as a thin shim around TS IR and does little "heavy lifting", the same cannot be said for other dialects in MLIR.
For example, the linalg dialect is designed to address the hierarchical optimization problem, wherein the goal is to enable code generation of efficient code or dispatch to existing, previously optimized, kernel code, without sacrificing ease of use and performance for either path.
Practically speaking, this entails representations of common mathematical operations, such as \mintinline{python}{matmul}, \mintinline{python}{conv}, and \mintinline{python}{batchnorm}, explicitly declaring semantics that are traditionally obtained only through compiler analysis (such as memory dependency) and transformations on such operations completely preserving such semantics.
We make extensive use of the linalg dialect as an intermediary between lower-level dialects, such as the affine and structured control flow dialects, and Torch dialect.
The structured control flow dialect is a straightforward formalization of control flow primitives, such as conditionals and loops, so we do not discuss it in great detail.
The affine dialect, on the other hand, provides a formalization of semantics that lend themselves to polyhedral compilation techniques\cite{polyhedral-mlir}, i.e., techniques that make dependence analysis and loop transformations efficient and reliable.

It's worth working through the lowering of one of the convolutional layers comprising BraggNN.
A PyTorch representation of this operation can be seen in Listing~\ref{lst:conv2dpy}, where \mintinline{python}{*_channel} parameters specify the multiplicity of the input and output feature maps, \mintinline{python}{kernel_size} refers to the characteristic dimension of the convolution filter/kernel, and \mintinline{python}{padding} refers to the amount of zero padding that should be added to the input.
\begin{longlisting}
	\inputminted{python}{sources/conv2d.py}
	\caption{PyTorch representation of 2D convolution.}
	\label{lst:conv2dpy}
\end{longlisting}
In terms of abstraction, it's important to note here that this is a completely declarative and abstract description of the operation, with neither specification nor constraint on how the data should be initialized or ordered in memory, nor the hardware that the computation should run on, nor whether some constituent computations could be (or should be) performed in parallel.

The first rung down the ladder of abstraction is TorchScript IR, as seen in Listing~\ref{lst:conv2dtorchscript}, where it's worth pointing out that the tensor corresponding to the weights of the convolution, namely~\mintinline{mlir}{%conv1.weight}, has a fully refined type (including data type, shape and striding) but neither the input tensor (\mintinline{mlir}{%x.1}) nor the output tensor (\mintinline{mlir}{%out.1}) do.
\begin{longlisting}
	\inputminted{mlir}{sources/conv2d.ts}
	\caption{TorchScript representation of 2D convolution.}
	\label{lst:conv2dtorchscript}
\end{longlisting}
This implies that the types of those tensors are determined at runtime, while, as already mentioned, these dimensions are necessary for further lowering.
Thus the input shape needs to be supplied by the user, while the output shape is in principle determined by the implementation of \mintinline{mlir}{aten::conv2d}.
This is reflected in the next rung, the first representation of 2D convolution as MLIR, as visible in Listing~\ref{lst:conv2dtorchmlir}.
\begin{longlisting}
	\inputminted{mlir}{sources/conv2d.torch.mlir}
	\caption{Torch dialect representation of 2D convolution.}
	\label{lst:conv2dtorchmlir}
\end{longlisting}
Note that here we have all types for all values fully refined, as well as having their value semantics\footnote{Value-type semantics means for an object that only its value is significant, not its identity. This necessarily implies that copies are performed by value and mutation isn't permitted.} emphasized by the \mintinline{mlir}{vtensor} type.
Aside from type refinement, the Torch dialect performs complex operator decomposition and translation to basic operations, as evident in the lowering to the linalg dialect (see Listing~\ref{lst:conv2dlinalg}), where we observe that \mintinline{mlir}{torch.aten.conv2d} is decomposed into output tensor declaration (\mintinline{mlir}{linalg.init_tensor}), output tensor initialization (\mintinline{mlir}{linalg.fill}) and finally actual convolution (\mintinline{mlir}{linalg.conv_2d_nchw_fchw}).
\begin{longlisting}
	\inputminted{mlir}{sources/conv2d.linalg.mlir}
	\caption{linalg dialect representation of 2D convolution.}
	\label{lst:conv2dlinalg}
\end{longlisting}
For a more involved example of this decomposition see the \commnt{appendix softmax}{appendix}.
One important thing to note at this level of abstraction is that the decomposition is as such in the service of preserving value semantics.
While value semantics are important for various analyses, they are in themselves an artifact of abstraction: if we can guarantee no aliasing of \mintinline{mlir}{%arg0} occurs downstream of this convolution, then the \mintinline{mlir}{linalg.init_tensor} and \mintinline{mlir}{linalg.fill} are wasteful (of memory and clock cycles).
More on this in the methodology section.
Note that here we also clearly see, for the first time, the sense in which MLIR is a family of mutually compatible dialects; \mintinline{mlir}{arith.constant} operations are part of the arith dialect, which is mutually compatible with most dialects in MLIR.

From the linalg representation there are two choices for the next level of abstraction:
\begin{enumerate}
	\item Affine Loops (affine dialect);
	\item Structured Control Flow Loops (scf dialect).
\end{enumerate}
Ultimately we will opt for only one of these (scf) but it's edifying to inspect both.
\begin{longlisting}
	\inputminted{mlir}{sources/conv2d.affine.mlir}
	\caption{affine dialect representation of a 2D convolution.}
	\label{lst:conv2daffine}
\end{longlisting}
As already mentioned, that the affine dialect is a precise formalization of polyhedral semantics.
It's worth emphasizing that syntax such as \mintinline{mlir}{affine_map<(d0, d1) -> (d0 + d1)>} (an \emph{affine map}, which represents the projection $y = d0 + d1$) is structured and computable.
That is to say, it has an object representation in memory during compilation that can be manipulated and queried.
For example, this representation is employed in proving the existence or absence of data dependencies between iterations of adjacent loop nests (such as those in Listing~\ref{lst:conv2daffine}) as a prerequisite for loop fusion; this is implemented by constructing the set of constraints on loop indices and memory accesses (i.e., \mintinline{mlir}{%1[%arg1, %arg2, %arg3, %arg4]} and \mintinline{mlir}{%arg0[%arg1, %arg5, %3, %4]}) and computing the feasible region (using either Presburger Arithmetic\cite{10.1145/3485539} or Fourier--Motzkin elimination\cite{10.2307/2322281}).
Note that this analysis abides a straightforward implementation exactly due to the explicit inclusion of the affine mapping between the loop iteration space and the memory accesses.
\begin{longlisting}
	\inputminted{mlir}{sources/conv2d.scf.mlir}
	\caption{scf dialect representation of a 2D convolution.}
	\label{lst:conv2dscf}
\end{longlisting}
The lowering to structured loops in Listing~\ref{lst:conv2dscf} illustrates the kinds of insights available from lowering the level of abstraction of an arbitrary DNN operation; while it's certainly straightforward to infer from the mathematical \commnt{maybe write down the equation?}{definition of convolution} that it is "embarrassingly parallel" across dimensions \commnt{double check}{(\mintinline{mlir}{batch_size, output_channels, output_height, output_width}),} this property is manifestly obvious when considering the resulting loop nest; note that in the inner body of the second loop nest, the load and stores from/to the ultimate result of the convolution (\mintinline{mlir}{%2}) only depend on (\mintinline{mlir}{%arg1, %arg2, %arg3, %arg4}) and that the store follows the load.
Thus, it becomes clear from just inspection that the inner loops (on \mintinline{mlir}{%arg5}, \mintinline{mlir}{%arg6}, \mintinline{mlir}{%arg7}) can be fully unrolled and that the resulting single loop can be parallelized across (\mintinline{mlir}{%c1} $\times$ \mintinline{mlir}{%c64} $\times$ \mintinline{mlir}{%c9} $\times$ \mintinline{mlir}{%c9}) workers.

The next step in the lowering is LLVM IR, an IR that is, technically speaking, not an MLIR dialect, even though there does exist an LLVM dialect and we do lower to LLVM dialect as an intermediary between MLIR and LLVM.
The purpose of further lowering to LLVM IR is to produce a representation of BraggNN that high-level synthesis tools can consume.
In particular, Xilinx's Vitis HLS, based on the Autopilot project\cite{Zhang2008}, recently enabled passing LLVM IR to the tool, rather than C/C++.
We Briefly discuss the role of HLS in the translation process.

\subsubsection{High-Level Synthesis and Down}\label{subsec:hlsdown}

High-level synthesis tools produce RTL descriptions of digital designs from high-level representations, such as C or C++\cite{10.1145/2514740, ferrandi2021bambu}, or in the case of Vitis HLS, LLVM IR.
Given a high-level representation, HLS fundamentally does three things in order to produce a corresponding RTL design:
\begin{enumerate}
	\item HLS schedules operations (such as \mintinline{mlir}{fmult}, \mintinline{mlir}{fadd}, \mintinline{mlir}{load}, \mintinline{mlir}{store}) in order to determine which such operations should occur during each clock-cycle. A schedule depends on three parameters:
	      \begin{itemize}
		      \item The topological ordering of the DFG/CFG (i.e., the dependencies of operations on results of other operations and resources);
		      \item The completion time for each operation;
		      \item The user's desired clock rate/frequency/cycle length.
	      \end{itemize}
	\item HLS associates high-level operations to particular RTL instantiations (called \emph{bindings}) for those operations;
	for example whether to associate an addition operation followed by a multiplication operation to two separate DSP instances, or whether to associate them both with a single DSP instance configured to compute fused-multiply-add;
	\item HLS builds an FSM that cycles (transitions from state to state) through the sequence of operations in the schedule (and extracts its RTL representation).
\end{enumerate}
The scheduling problem can be expressed as an integer programming problem\cite{tuprints9272} whose solution in general, as all for all integer programming problems, is NP-Hard (NP-complete even).
%\begin{figure}[H]
%	\includegraphics[width=\columnwidth]{figures/schedule}
%	\caption{Scheduling problem for HLS.}
%	\label{fig:schedule}
%\end{figure}
For certain instances, the scheduling problem can be formulated as a system of difference constraints.
%\begin{figure}[H]
%	\includegraphics[height=\columnwidth]{figures/sdc_constraints}
%	\caption{System of difference constraints formulation for the scheduling problem.}
%	\label{fig:sdc_constraints}
%\end{figure}
%\begin{figure}
%	\includegraphics[width=\columnwidth]{figures/sdc_graph}
%	\caption{This is a placeholder.}
%	\label{fig:sdc_graph}
%\end{figure}

Such a formulation induces a unimodular matrix representation of the constraint system which is amenable to Bellman-Ford $O(n^2 + mn)$\commnt{this is phrased poorly - unimodular guarantees you integral solutions to the LP but also gives you the shortest path analogy property}{ for checking consistency of a solution (essentially the shortest path problem) }and an LP approach for optimizing the critical path\cite{1688836}.
%\begin{figure}
%	\includegraphics[width=\columnwidth]{figures/unimodular}
%	\caption{This is a placeholder.}
%	\label{fig:unimodular}
%\end{figure}

In addition to fulfilling these three fundamental tasks, High-level synthesis tools such as Vitis, Bambu\cite{ferrandi2021bambu}, LegUp\cite{10.1145/2514740} perform standard compiler optimization passes on the IR (that they either receive or produce internally).
Optimization passes such as store-load forwarding, common subexpression elimination, and constant propagation.
Loop-unrolling and tiling are also performed at the behest of the user.
Ostensibly, after running the LLVM IR corresponding to BraggNN through e.g. Vitis, we should have a RTL description of a digital design that implements the functionality of BraggNN, including control flow, floating point operations (potentially, by way of DSPs), and memory I/O.
At this level of abstraction, there remain two more steps prior to being able to actually deploy to an FPGA; one of them being a final lowering, so called logic synthesis, and the other being Place and Route (PnR).

Logic synthesis is the process of mapping RTL to actual hardware primitives on the FPGA (so called \emph{technology mapping}), such as lookup tables (LUTs), registers, and DSPs (in cases where they haven't been explicitly instantiated).
Logic synthesis produces a network list (netlist) describing the connectivity of various parts of the design.
For example,
\begin{longlisting}
	\inputminted{verilog}{sources/always.v}
	\caption[Long Code Example]{A long code example which will break across pages.}
	\label{lst:long}
\end{longlisting}
\noindent corresponds to a state of an FSM during which the registers \mintinline{verilog}{reg_1608}, \mintinline{verilog}{reg_1613}, \mintinline{verilog}{reg_1618} are updated with new values from wires (\mintinline{verilog}{fu_1076_p2}, \mintinline{verilog}{notrhs18_fu_1082_p2}, \mintinline{verilog}{fu_646_p2}).
Assuming these registers are updated in other states from differing wires, this logic will synthesize to muxs and possibly mux trees (depending on how many different input wires feed these same registers).
Such muxs (and mux trees) are actually implemented using LUTs (more on this in Section~\ref{subsec:parallel-toposort-scheduling}).
Another example is the implementation of floating point units in terms of DSPs; depending on user parameters and/or design analysis, DSP resource consumption for floating point multiplication and addition can different greatly (more on this in the methodology section).

Finally, after the netlist has been produced, the entire design undergoes PnR.
The goal of PnR is to determine which configurable logic block within an FPGA should implement each of the units of logic required by the digital design.
PnR algorithms need to minimize distances between related units of functionality (in order to minimize wire delay), balance wire density across the entire fabric of the FPGA (in order to reduce route congestion), and maximize the clock speed of the design (a function of both wire delay, logic complexity, and route congestion).
The final, routed design, can then be deployed to the FPGA by producing a proprietary \emph{bitstream}, which is flashed to the FPGA.

In general, both of these final steps (logic synthesis and PnR) can only be performed by the proprietary tools of the hardware manufacturers (e.g., Vivado by Xilinx) and thus, from our perspective their inner workings are completely unknown.
Recently, open source alternatives for certain FPGAs have become available, thanks to herculean efforts made to reverse engineer the various bitstream formats of, for example, some of Xilinx's architectures\cite{6546003}, and reimplement logic synthesis in open source.
Namely, \cite{wolf2013yosys} is a framework for Verilog RTL synthesis.
It provides a basic set of synthesis algorithms for mapping to Xilinx and Lattice FPGA (as well as ASIC standard cells).
We use Yosys as a basis of comparison for commercial tools and as a way to investigate the limitations of those tools (i.e., when Vivado fails, we can reason by analogy with Yosys, why it might've failed).

\section{Methodology}\label{sec:methodology}
Inspired by the FloPoCo~\cite{8877424} project, our design methodology can best be described as \emph{computing just right}.
We aim for just the right amount of compute and abstraction for the task at hand, no more, no less.
This methodology stands in contrast to the design methodologies of architecture designers for commodity hardware accelerators (such as GPUs), which must support a large set of use cases.
Accordingly, our methodology consists of five techniques distilled from such general purpose architectures and tools but specialized for our specific purposes.

\subsection{Abstract Interpretation for Efficient Transformations}\label{subsec:loop-unrolling}

It is well known that the most straightforward way to achieve lowest latency inference of a DNN is to linearize the control flow graph~\cite{osti_1574050} (i.e., remove branches) and flatten the dataflow graph as much as possible~\cite{10.1145/3295500.3356173} (i.e., execute as many operations in parallel as possible).
For intermediate level representations of a DNN, this corresponds to loop unrolling followed by fusion (alternatively known as unroll and jam~\cite{thomas1971catalogue}).
For example, consider the pair of loop nests in Listing~\ref{lst:loop_fusion}, corresponding to \inlinepython{Conv2d(64, 1, 3)}.
General purpose compilers can readily unroll each of the loop nests, but prior to fusion, due to their conservative correct guarantees, are bound to verify that memory independence constraints are satisfied for all pairs of stores and loads in each of the loop nests.
Thus, after unrolling the inner loops of the second loop nest (on \mintinline{mlir}{%arg5}, \mintinline{mlir}{%arg6}, \mintinline{mlir}{%arg7}) we incur
\[
	\mathtt{\%c1} \times \mathtt{\%c3} \times \mathtt{\%c3} \times 2 \times 4
\]
memory independence checks.
Consequently, unroll and fuse optimizations take increasingly longer as one experiments with larger and larger unroll thresholds; the \emph{unroll threshold} determines which loops will be fully unrolled (all loops with trip count less than or equal to the threshold will be unrolled).
\begin{longlisting}
	\inputminted[highlightlines={5,6,17,18,19,22},frame=lines]{mlir}{sources/loop_fusion.mlir}
	\caption{Loop fusion and unrolling example, for loops corresponding to \inlinepython{Conv2d(64, 1, 3)}, with \hl{emphasis} on loads and stores whose independence must verified.}
	\label{lst:loop_fusion}
\end{longlisting}
\begin{longlisting}
	\inputminted[highlightlines={5,6,11-13,16},linenos=true,frame=lines,numbersep=\mintednumbersep]{mlir}{sources/unrolled_loop.mlir}
	\caption{Unrolled and fused loop-nest, for loops corresponding to \inlinepython{Conv2d(1, 64, 3)}, with \hl{emphasis} on which store-load forwards can be performed.}
	\label{lst:storeloadforwarding}
\end{longlisting}

Following unroll and jam, we are able to perform \emph{store-load forwarding}, i.e., we are able to promote those store operations which are subsequently loaded from (with no intervening stores) to registers.
For example, consider the above fused loop nests, with the second loop nest having the inner three loops fully unrolled (see Listing~\ref{lst:storeloadforwarding}); the store operations on lines 6 and 13 can be entirely eliminated and the load operation on line 5 can be forwarded to the \inlinemlir{arith.addf} on line 15.
Note, for each load operation that follows a store operation, a compiler must check whether the store and the load access the same location in memory, and further verify that there are no intervening store operations to the same memory address.
In general, this requires solving a system of constraints~\cite{10.2307/2322281}.
We observe that as loops are further and further unrolled, the cost of this particular optimization grows polynomially; consider a fully unrolled loop nest, with many parallel dataflows, for which a store operation in one dataflow might be forwarded across a parallel dataflow (and thus incur checks against all stores and loads in that parallel dataflow).

In principle, we might rely on MLIR or LLVM to perform each of these optimization passes.
The chief impediment to relying on these general purpose compilers for our needs is the runtime complexity of the their implementations.
For arbitrary programs this development time cost moderate, especially given that most development is done without these optimization (leaving the aggressive optimizations for release builds).
For us, given that the logic and dataflow of BraggNN is fixed (having already been iterated on), and given that we are in fact searching the design space for optimal low-level representations of the DNN, the runtimes of these optimization passes are prohibitive (taking on the order of hours and sometimes even days to complete).
Moreover, often their rigor and conservatism are unnecessary given a high-level understanding of the structure of BraggNN; for example, in the case of fully unrolling the above loop nests and forwarding from the stores during initialization to the loads during accumulation, the region within which the forward is safe is clear from the semantics of convolutions.
The loop indices and corresponding memory addresses for these safe store-load forwards are simple to compute analytically and ahead of time (even in the presence of complications such as strided tensors).

In order to avoid the runtime costs associated with these conservative optimization passes, we implement an abstract interpreter for sufficiently lowered DNNs.
Concretely, we lower BraggNN to the structured control flow (\inlinemlir{scf}) dialect and then interpret this representation of BraggNN with alternative semantics.
Firstly, our interpreter evaluates functions of loop indices, such as 
\begin{minted}[autogobble,xleftmargin=0.2\columnwidth,xrightmargin=0.2\columnwidth]{mlir}
	#map = affine_map<(d0, d1) -> (d0 + d1)>
	%3 = affine.apply #map(%arg3, %arg6)
\end{minted}
or 
\begin{minted}[autogobble,xleftmargin=0.2\columnwidth,xrightmargin=0.2\columnwidth]{mlir}
	%3 = arith.addi %arg3, %arg6
\end{minted}
where \inlinemlir{%arg3}, \inlinemlir{%arg6} are loop indices.
This enables us to determine array indices of stores and loads, such as 
\begin{minted}[autogobble,xleftmargin=0.2\columnwidth,xrightmargin=0.2\columnwidth]{mlir}
	%2 = memref.alloca() : memref<8xf16>
	%c1 = arith.constant 1.0
	memref.store %c1, %2[%3]
\end{minted}
Note that evaluation of such memory index arithmetic is typically deferred to runtime in conventional accelerators because of either the control flow inherent in a DNN, or simply a lack of available registers (to store the results).
Thus, we are able to precompute these indices, saving cycles, because BraggNN lacks control flow and FPGAs are abundant in registers.
Note also that we do not evaluate values corresponding to \inlinemlir{memref.load}s, as they represent BraggNN weights or activations; our interpreter implements such values as proxy objects and merely records the arithmetic operations performed on them.

Secondly, our interpreter unrolls loops by executing them while enforcing SSA.
That is to say, for a loop whose body has repeated assignments to the same value (ostensibly violating SSA), such as 
\begin{minted}[autogobble,xleftmargin=0.2\columnwidth,xrightmargin=0.2\columnwidth]{mlir}
	%c0 = arith.constant 0
	%c1 = arith.constant 1
	%c3 = arith.constant 3
	%c5 = arith.constant 1.35
	scf.for %arg1 = %c0 to %c3 step %c1 :
		// cast int to fp
		%2 = arith.sitofp %arg1
		%3 = arith.addf %arg1, %c5
\end{minted}
we execute the loop and instantiate unique identifiers for the result of each operation:
\begin{minted}[autogobble,xleftmargin=0.2\columnwidth,xrightmargin=0.2\columnwidth]{mlir}
	%c0 = arith.constant 0 
	%c1 = arith.constant 1 
	%c2 = arith.constant 2 
	%c3 = arith.constant 3 
	%c5 = arith.constant 1.35
	%2_0 = arith.sitofp %c0
	%3_0 = arith.addf %2_0, %c5
	%2_1 = arith.sitofp %c1
	%3_1 = arith.addf %2_1, %c5
	%2_2 = arith.sitofp %c2
	%3_2 = arith.addf %2_2, %c5
	%2_3 = arith.sitofp %c3
	%3_3 = arith.addf %2_3, %c5
\end{minted}
This enables us to both unroll the loop and track dataflow through arithmetic operations (see Section~\ref{subsec:parallel-toposort-scheduling}).
Finally, our interpreter reinterprets \inlinemlir{memref}s as \emph{geometric symbol tables} (i.e., symbol tables indexed by array indices rather than identifiers/names) and stores/loads as assignments/reads to/from those symbol tables.
Such semantics, in combination with fully evaluated array indices, enable us to simultaneously track dataflow through activation buffers (which appear as \inlinemlir{memref}) and perform store/load forwarding.

The dataflow analysis carried out by our interpreter enables us to easily infer the flow of weights through the DNN and thus enables us to experiment with two weight storage strategies; namely we can either store weights as BRAMs, uniquely associated with a set of DSP blocks, or as a collection of free registers. 
The dataflow analysis also enables us to identify sequences of multiplications and additions and group them together such that they can be scheduled to reuse accumulator registers associated with floating point block instantiations (effectively forming a multiply-accumulator); 
indeed, this grouping is the chief optimization that enables us to efficiently map BraggNN to FPGA.
See Section~\ref{subsec:parallel-toposort-scheduling} for a discussion on the advantages of both of these.

% We also reinterpret remaining stores and loads to memory as reads and writes to registers, thus simplifying our design; such stores and loads would otherwise translate to stores and loads from Block RAM (BRAM).
% Furthermore, since BraggNN is a relatively small DNN, we inline absolutely all of the weight tensors as constants and perform \emph{constant propagation}.
% In summary our abstract interpreter performs the following transformations on the \inlinemlir{scf} representation of BraggNN:
% \begin{enumerate}
% 	\item We completely eliminate any latency due to loading from or storing to BRAM for intermediate activations;
% 	\item We completely eliminate all logic (i.e., integer arithmetic) related to calculating memory offsets, a non-trivial reduction in instruction count and design complexity (see figure <figure> for reduction in instruction count);
% 	\item We instantiate a reduced set of floating point operation cores, and thus reduce complexity and overall latency of our design.
% \end{enumerate}
% Note, we are able to perform the memory to registers promotions owing to the fact that BraggNN is a relatively compact DNN and our target FPGA is plentiful in registers.
% The reinterpreted semantics implemented by our abstract interpreter are presented in Eqns~\eqref{eqn:semantics}.

% \begin{figure}
% 	% \begin{equation}\label{eqn:semantics}
% 	% 	\begin{split}
% 	% 		\llbracket \%\texttt{\small var} = \inlinemlir{memref.alloca}\texttt{()}\!\!:\!\inlinemlir{memref}\!\!<\!\!k\inlinemlir{xf16}\!\!> \rrbracket &= [\text{allocate }k\text{ 16 bit registers}] \\
% 	% 		\llbracket \mintedinline{mlir}{\%5 = memref.load \%arg0[\%arg1, \%arg5, \%3, \%4]} \rrbracket &= [\text{allocate }k\text{ 16 bit registers}] \\
% 	% 	\end{split}
% 	% \end{equation}
% 	\begin{multline*}
% 		\llbracket \inlinemlir{$\%$var = memref.alloca() : memref<$k$xf16>} \rrbracket \implies \\
% 			\text{allocate $k$ 16 bit registers}
% 	\end{multline*}
% 	\begin{multline*}
% 		\quad\llbracket \inlinemlir{$\%$5 = memref.load $\%$var[$\%$m]} \rrbracket \implies \text{read $m$th register} \quad
% 	\end{multline*}
% 	\begin{multline*}
% 		\quad\quad\llbracket \inlinemlir{$\%$8 = arith.mulf $\%$5, $\%$6} \rrbracket \implies \text{ $\%8 = \%5 \times \%6$} \quad\quad\quad
% 	\end{multline*}
% 	\begin{multline*}
% 		\quad\llbracket \inlinemlir{memref.store $\%$8, $\%$var[$\%$m]} \rrbracket \implies \text{store $\%$8 in $m$th register}\quad
% 	\end{multline*}
% 	\begin{multline*}
% 		\llbracket \inlinemlir{scf.parallel (...) = (...) to (...) step (...)} \rrbracket \implies \\
% 			 \text{store $\%$8 in $m$th register}
% 	\end{multline*}

% 	\caption{This is a placeholder.}\label{fig:semantics}
% \end{figure}

\subsection{Parallel Topological Sort Scheduling}\label{subsec:parallel-toposort-scheduling}

In addition to reducing the runtime of the compiler frontend, our abstract interpreter simplifies the representation such that we may emit a simplified LLVM IR to pass to downstream tools (such Vitis HLS), and thus, in theory reduce their runtime as well.
In actuality, performing these optimizations has no effect on the runtime of Vitis HLS, since it will rerun the same (or similar) passes as part of its transformation pipeline.
Thus, a fundamental challenge in implementing BraggNN as a digital design is reducing the time taken by Vitis HLS in performing its own optimizations; in addition to those ultimately redundant optimizations, Vitis' HLS scheduling algorithms takes an inordinate amount of time to schedule the large number of operations that comprise BraggNN when fully unrolled, even after eliding unnecessary operations (such as the integer arithmetic associated with computing memory offsets).

Thus it becomes necessary to completely eliminate Vitis and (therefore HLS) from the design process.
Recall that the critical function which Vitis HLS fulfills is the scheduling of operations during each clock cycle, in such a way that they respect the dataflow graph of BraggNN; that schedule then informs the construction of a corresponding FSM.
In general, for complex programs, this is indeed a computationally intensive task, necessitating formulating the scheduling problem and solving it using either an integer programming solver or, if the problem is formulated as an SDC, an LP solver.
In the case of BraggNN, and in fact many DNNs, where the dataflow is very regular and where parallel execution is only bounded by the number of DSPs (assuming BRAM access is eliminated), it is straightforward to construct the optimal schedule based solely on a topological sort of the operations.

In fact, by computing the schedule ``by hand'', we can exercise more precise control over DSP usage and thus further reduce the overall complexity of the design.
That it to say, we can schedule using the maximum number of DSPs possible during each state of the FSM, in contrast to Vitis HLS, which attempts to make conservative use of DSPs (which leads to excessive LUT usage to support time multiplexing of the DSPs); see figure <...> for a comparison the LUT usage using our scheduling algorithm versus Vitis HLS scheduling algorithm.
One thing to note here is that computing a schedule even just using sequential topological sort becomes costly (in terms of runtime) for the larger scalings of BraggNN, where we need to schedule upwards of 1E6 operations.
To mitigate these costs as well (in order to enable fast iteration on design choices) we implement a parallelized topological sort~\cite{sanders2019sequential}.
See Algorithm~\ref{alg:toposort} for a specification of our scheduling algorithm.

\begin{algorithm}
	\caption{Placeholder}\label{alg:toposort}
	\begin{algorithmic}
		\Require $n \geq 0$
		% \Ensure $y = x^n$
		% \State $y \gets 1$
		% \State $X \gets x$
		% \State $N \gets n$
		% \While{$N \neq 0$}
		% \If{$N$ is even}
		% \State $X \gets X \times X$
		% \State $N \gets \frac{N}{2}$  \Comment{This is a comment}
		% \ElsIf{$N$ is odd}
		% \State $y \gets y \times X$
		% \State $N \gets N - 1$
		% \EndIf
		% \EndWhile
	\end{algorithmic}
\end{algorithm}

\subsection{Bit Twiddling Hacks}\label{subsec:bit-twiddling-hacks}

FPGAs do not permit dynamic reconfiguration of DSPs to support fully heterogeneous operations; a set of DSPs associated with floating point addition can be dynamically reconfigured to perform subtraction but multiplication (likewise vice-versa).
Thus, in order to maximize reuse of DSPs we only instantiate floating point IP cores for multiplication.
We map the remaining operations necessitated by BraggNN (\inlinemlir{fsub}, \inlinemlir{fexp}, \inlinemlir{relu}, \inlinemlir{fdiv}) to (\inlinemlir{fmul}) and addition (\inlinemlir{fadd}).
For the cases of \inlinemlir{fsub} and \inlinemlir{fexp} this is straightforward (a bit flip on the sign bit to handle the former and a Taylor series expansion to handle \inlinemlir{fexp}).
For the case of \inlinemlir{relu}, note that for a IEEE 754 $n$-bit floating point number $x$
$$
\max(0, x) = x \iff x[0] = 0
$$
where $x[0]$ represents the sign bit of $x$.
Division is the only primitive operation that presents a serious challenge to normalization in this way.
To represent division (in terms of \inlinemlir{fmul}) we exploit the fact that aliasing a floating point number as an integer effectively calculates the approximate binary logarithm~\cite{enwiki:1081681080} of the number;
we use this property to approximate the inverse (of a floating point number) and then perform division through multiplication by that inverse.
In our experiments (and prior work~\cite{10.1007/978-0-387-72258-0_14}) this approximation incurs approximately a $4\%$ difference in accuracy per division.

Finally, we experiment with alternative bitwidth implementations of IEEE754 floating point.
With respect to the BraggNN training, we observe that the sample data does not use a full 8 bit exponent (see Figure~\ref{fig:numexp}).
\begin{figure}
	\includegraphics[width=\columnwidth]{figures/exp_hist}
	\caption{Range of exponent values for BraggNN weights.}\label{fig:numexp}
\end{figure}
With this in mind, we deploy BraggNN using half precision floats, i.e., using 5 bits to represent the exponent and 11 bits to represent the mantissa.
This produces floating point arithmetic cores that use fewer registers, LUTs, and having smaller wire delays, again leading to a reduction in overall complexity and end to end latency.


\section{Evaluation}\label{sec:evaluation}
We evaluate our design methodology, as applied to BraggNN, and a few NN functional units, against ScaleHLS\cite{ye2021scalehls} and SODA-OPT\cite{9516615}, two state-of-the-art but general purpose DNN compiler to HLS tool flows.
Both tools perform design space exploration but at the cost of high-runtimes and ultimately inferior results \emph{when latency is the only metric}.

\section{Acknowledgements}

% \todo


\bibliographystyle{ACM-Reference-Format}
\bibliography{ref}


\end{document}
