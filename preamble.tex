\usepackage[utf8]{inputenc}
\usepackage{url}
\usepackage{color,fancyvrb}
 \usepackage[frozencache,cachedir=_minted]{minted}
% \usepackage[finalizecache,cachedir=_minted]{minted}
%\usepackage[cachedir=_minted,outputdir=latex_output]{minted}
%\usepackage[cachedir=_minted]{minted}
\renewcommand{\MintedPygmentize}{./scripts/pygmentize.py}

\newenvironment{longlisting}{\captionsetup{type=listing}}{}
%\usepackage[skip=2pt]{caption} % example skip set to 2pt
\setlength{\belowcaptionskip}{10pt plus 3pt minus 2pt} % Chosen fairly arbitrarily
\setlength{\abovecaptionskip}{10pt plus 3pt minus 2pt} % Chosen fairly arbitrarily

\setminted{
	% linenos=true,
	autogobble,
	breaklines,
	tabsize=2,
	fontsize=\footnotesize,
	encoding=utf8,
	frame=lines,
	fontfamily=zi4
}
\usepackage{etoolbox,xpatch}

\makeatletter
\AtBeginEnvironment{minted}{\dontdofcolorbox}
\def\dontdofcolorbox{\renewcommand\fcolorbox[4][]{##4}}
\xpatchcmd{\inputminted}{\minted@fvset}{\minted@fvset\dontdofcolorbox}{}{}
\xpatchcmd{\mintinline}{\minted@fvset}{\minted@fvset\dontdofcolorbox}{}{} % see https://tex.stackexchange.com/a/401250/
\makeatother

\setcopyright{acmlicensed}
\acmDOI{00.000/000_0}
\acmISBN{000-0000-00-000/00/00}
\acmConference[ABC '00]{A Boilerplate Conference}{January 0000}{Norfolk, Virginia, USA}
\acmYear{0000}
\copyrightyear{0000}
\acmPrice{0.00}

% When "techreport" class parameter is supplied:
%   * hide reference block from the front page
%   * hide copyright note from the front page
%   * change conference info in the header of alternate pages to title
%   * add page number in the center of the footer of non-title pages
\makeatletter
\@ifclasswith{acmart}{techreport}{
	\settopmatter{printacmref=false}
	\renewcommand\footnotetextcopyrightpermission[1]{}
	\fancyhead{}
	\fancyfoot{}
	\fancyhead[L]{\shorttitle}
	\fancyhead[R]{\shortauthors}
	\fancyfoot[C]{\thepage}
	% \AtBeginDocument{The pagesize is \texttt{a5paper}.\par}
}{}
\makeatother



\newif\iffinal
\finalfalse


\iffinal
	\newcommand{\maxx}[1]{}
	\newcommand{\ryan}[1]{}
	\newcommand{\todo}[1]{}
	\newcommand{\commnt}[2]{#2}
\else
	\newcommand{\maxx}[1]{{\textcolor{red}{ Max: #1 }}}
	\newcommand{\ryan}[1]{{\textcolor{magenta}{ Ryan: #1 }}}
	\newcommand{\todo}[1]{{\textcolor{red}{ TODO: #1 }}}
	\newcommand{\commnt}[2]{{{\color{green} #1} {\color{blue} #2}}}
\fi
